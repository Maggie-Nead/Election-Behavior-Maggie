\PassOptionsToPackage{unicode=true}{hyperref} % options for packages loaded elsewhere
\PassOptionsToPackage{hyphens}{url}
%
\documentclass[]{article}
\usepackage{lmodern}
\usepackage{amssymb,amsmath}
\usepackage{ifxetex,ifluatex}
\usepackage{fixltx2e} % provides \textsubscript
\ifnum 0\ifxetex 1\fi\ifluatex 1\fi=0 % if pdftex
  \usepackage[T1]{fontenc}
  \usepackage[utf8]{inputenc}
  \usepackage{textcomp} % provides euro and other symbols
\else % if luatex or xelatex
  \usepackage{unicode-math}
  \defaultfontfeatures{Ligatures=TeX,Scale=MatchLowercase}
\fi
% use upquote if available, for straight quotes in verbatim environments
\IfFileExists{upquote.sty}{\usepackage{upquote}}{}
% use microtype if available
\IfFileExists{microtype.sty}{%
\usepackage[]{microtype}
\UseMicrotypeSet[protrusion]{basicmath} % disable protrusion for tt fonts
}{}
\IfFileExists{parskip.sty}{%
\usepackage{parskip}
}{% else
\setlength{\parindent}{0pt}
\setlength{\parskip}{6pt plus 2pt minus 1pt}
}
\usepackage{hyperref}
\hypersetup{
            pdftitle={Research Question},
            pdfauthor={Maggie Nead},
            pdfborder={0 0 0},
            breaklinks=true}
\urlstyle{same}  % don't use monospace font for urls
\usepackage[margin=1in]{geometry}
\usepackage{graphicx,grffile}
\makeatletter
\def\maxwidth{\ifdim\Gin@nat@width>\linewidth\linewidth\else\Gin@nat@width\fi}
\def\maxheight{\ifdim\Gin@nat@height>\textheight\textheight\else\Gin@nat@height\fi}
\makeatother
% Scale images if necessary, so that they will not overflow the page
% margins by default, and it is still possible to overwrite the defaults
% using explicit options in \includegraphics[width, height, ...]{}
\setkeys{Gin}{width=\maxwidth,height=\maxheight,keepaspectratio}
\setlength{\emergencystretch}{3em}  % prevent overfull lines
\providecommand{\tightlist}{%
  \setlength{\itemsep}{0pt}\setlength{\parskip}{0pt}}
\setcounter{secnumdepth}{0}
% Redefines (sub)paragraphs to behave more like sections
\ifx\paragraph\undefined\else
\let\oldparagraph\paragraph
\renewcommand{\paragraph}[1]{\oldparagraph{#1}\mbox{}}
\fi
\ifx\subparagraph\undefined\else
\let\oldsubparagraph\subparagraph
\renewcommand{\subparagraph}[1]{\oldsubparagraph{#1}\mbox{}}
\fi

% set default figure placement to htbp
\makeatletter
\def\fps@figure{htbp}
\makeatother


\title{Research Question}
\author{Maggie Nead}
\date{6/11/2020}

\begin{document}
\maketitle

Research Question Write Up

\begin{verbatim}
 •  Introduction: 
 Congress: The Electoral Connection written by David Mayhew refers to Members of Congress as “single-minded seekers of reelection” stating that in order for Members of Congress to reach their goals of moving up the ranks, pushing policy goals, etc. they must first be in congress. This in turn makes reelection the single focus of current members. This research seeks to address the question of what happens when the goal of reelection is taken of the table.

 •  Questions:
     o  Is the behavior between members who get reelected different from members who do not?
           How?
     o  Are members of congress not seeking reelection more willing to step outside of party lines on policy than those seeking reelection?
     o  Are corporate interests less important to members not seeking reelection?
           Or are members who do not advocate for corporate interests less likely to be reelected?

•   Has this changed over time?

•   Why it is important

•   Thesis

•   Data
For the purposes of this research reelection will be defined as any member who is in consecutive congresses. Members who were not reelected shall be defined as any member who shows up in one congress and not the following congress. This research may further break this down between members who ran for reelection and lost and those who did not run for reelection at all.

•   References
  o Mayhew, David. 1974. Congress: The Electoral Connection. New Haven: Yale University Press.
\end{verbatim}

\end{document}
